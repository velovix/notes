\documentclass{article}
\author{Tyler Compton}
\title{Chapter 1 - Introduction}

\usepackage{float}

\begin{document}

\maketitle
\tableofcontents

\section{Introduction}
What is an operating system? The concept is hard to define. Any program that
comes with the computer is an approximation. The operating system definitely
includes the kernel, a program that is always running. It is in charge of
managing devices, which is the basic tasks that operating systems need to
accomplish.

A bootstrap program is the program that starts the kernel early on in the boot
process. Usually stored in a ROM or EPROM. Called firmware. Traditionally, a
virus would infect the computer, then write to the bootstrap program to make it
execute a rouge kernel instead of the legitimate one. This was very hard to get
rid of, because you couldn't boot up your computer wihtout it going to the
rouge kernel.

\section{Computer System Organization}

The CPU is where instructions are executed. Memory or RAM is where instructions
and data are stored. Non-volatile memory, AKA a hard drive or solid state disk
is for data that needs to persist between reboots, like the kernel.

\section{Managing the Needs of Multiple Devices}
These various devices may need attention from the kernel at any given time. How
can we find out which ones need attention?

\subsection{Polling}
Ask each device if it needs attention in order, when the last device is hit, go
back to the first. This method is simple, but inefficient. The kernel ends up
spending a lot of time accomplishing nothing.

\subsection{Interrupts}
The device sends an alert through the interrupt vector, which has addresses of
all the service routines. A trap or exception is a software-generated interrupt
for requests or errors. This is the way modern operating systems work. If two
devices need attention at the same time, the kernel must choose intelligently
which one will be addressed first.

Critical interrupts shouldn't be happening all the time, so the CPU has a lot
of downtime. The CPU needs this downtime to execute and switch between
programs.

\section{Storage Device Hierarchy}

Storage falls into a natural hierarchy based on read/write speed and storage
capacity. Below is a typical example of a storage device hierarchy on a modern
machine.

\begin{enumerate}
	\item Registers
	\item Cache
	\item Main Memory
	\item Hard Disk
	\item Optical Disks
\end{enumerate}

\section{Caching}

Caching is an important way to speed up the storage device hierarchy,
specifically when reading from slower storage devices. A commonly accessed
resource that lives in a slower storage device can be stored temporarily in a
faster storage device. This can speed up access to that resource significantly,
which is helpful if that resource is accessed frequently.

An example would be Netflix. A popular movie may streamed quite a few times at
once. It would be costly to have to consult the hard disk every time someone
tries to stream that movie. So, Netflix will store these popular movies in a
cache, oftentimes in RAM, which can be accessed much faster.

Caching is a tough programming problem, because the progarm has to be able to
intelligently figure out what should be cached. This can be based off of
frequency of use. However, if a resource is constantly changing, it can be hard
to keep the cache updated. Dirty caches can result poor user experiences
because the program doesn't seem to be reacting to user input, even though it
is. It can be difficult to notify a cache that it needs to be updated,
especially if the cache is being stored remotely, like in a user's system.

\section{Multicore Processors}

Traditionally, the computer would only have one general purpose processor and
potentially a few special purpose processors. However, multicore processors are
becoming increasingly common. Each core of a multicore processor is itself its
own processor that has its own register and cache, but all share access to
other computing devices, like storage.

Multicore processors have the ability to do multiple things truly at once, but
introduce new challenges. Access to shared resources must be orchestrated
carefully and processes have to be shared across processors in an efficient
way.

On a higher level, distributed systems are becoming increasingly common as
cloud software solutions. Instead of having a task executed by one computer
with multiple pyrocessing cores, tasks can be shared among multiple computers.
This allows many tasks to be to be executed quickly and at once. The same
and more problems apply to these systems that apply to multicore processors.

\section{Kernels}

Even though the kernel is the authority on device access, it must give programs
access to the processor. This access is still restricted, however. Processors
have different modes that give programs varying amounts of access. The kernel
gives access to programs in user mode, so that programs cannot completely
destroy the computer.

Basic programs are executed in user space. However, direct access to devices is
done in kernel space. User programs can request the kernel to act on a device
using system calls. When a program makes a system call, execution is
transferred to the kernel, the operation is performed, and execution is
transferred back to the user program.

\section{Virtualization}

Operating systems can be executed within other operating systems using a
program that provides a fake environment that a guest operating system can run
in. The guest operating system should have no idea that it is being run as a
guest.

A guest operating system can be built for a different archetecture than the
host's, but doing so requires an emulator. Emulators are generally the slowest
method of virtualization because the emulator needs to convert commands of one
type to another.

Quicker virtualization can be done if both the guest and host operating systems
use the same archetecture. Programs being executed in the guest environment can
have commands executed more directly by the host processor.

\end{document}

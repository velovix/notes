\title{Chapter 15: Security}
\author{Tyler Compton}

\documentclass{article}

\begin{document}

\maketitle
\tableofcontents

\section{Introduction}

A system is secure if all resources are accessed as intended under all
circumstances. Intruders are people who look to breach your security and
perform unintended operations. Attacks can be done either by accident or
maliciously. It is generally easier to prevent accidental attacks.

\section{Security Violation Categories}

\subsection{Breach of Confidentiality}

A breach of confidentiality occurs when a user becomes able to read data that
they are not authorized to access. Privacy concerns arise oftentimes because
of the possibility of these breaches.

\subsection{Breach of Integrity}

A breach of integrity occurs when a user is able to modify data that they are
not authorized to modify.

\subsection{Breach of Availability}

A breach of availability occurs when a user is able to destroy data that they
are not authorized to destroy.

\subsection{Theft of Service}

A theft of service occurs when a user is able to abuse the resources the
service provides.

\subsection{Denial of Service}

A denial of service occurs when a user is able to prevent others from
legitimately using the software.

\section{Security Violation Methods}

There are many kinds of ways a user can violate security.

\subsection{Masquerading}

A user can potentially pretend to be a user with more access privilages and
take actions they would not normally be able to take on that user's behalf.

\subsection{Replay Attack}

A user can repeat a legitimate message an illegitimate amount of times. For
instance, if the user forced their client to send multiple authorized requests
for money, this would be a replay attack because it would allow the user to get
the money more times than intended.

\subsection{Man-in-the-middle Attack}

A hacker puts itself somewhere in between a client and server or between
servers to intercept and modify the data being transferred.

\subsection{Session Hijacking}

A user can steal the session information of another user, making the server
think that the hacker is authenticated as the victim. Operations can then be
done on their behalf.

\section{Security Measure Levels}

It is impractical to have a piece of software with absolute security, but steps
can be taken to make it difficult enough to breach security that it becomes
impractical.

Security must be implemented at four levels.

\subsection{Physical}

The data centers, servers, and terminals connected to these services must not
be publically available.

\subsection{Human}

The people who do have access to these physical devices must be trusted. Steps
need to be taken to avoid social engineering, phishing, and dumpster diving, to
name a few.

\subsection{Operating System}

The operating system should have protection mechanisms in place to protect
itself from intrusion.

\subsection{Network}

The network must be built to avoid man-in-the-middle attacks and ways hackers
might try to bring the system down and deny service.

\section{Program threats}

\subsection{Trojan Horse}

A trojan horse is a generally small piece of software that misuses its
environment. It exploits mechanisms in order to allow programs written by one
user to be run by another user. Trojan horses are responsible for the lion's
share of spam attacks.

\subsection{Trap Door}

A trap door is an exploit intentionally put into a piece of software. It
provides a special way to subvert normal security protocol and gives a way for
other software to abuse the program that the trap door has been embedded in.
An untrusted compiler might inject trap doors into its software.

\subsection{Logic Bomb}
A logic bomb is a section of code that, when executed, does some operation that
is harmful in some way to the program.

\subsection{Stack and Buffer Overflows}

Stack and buffer overflows abuse a bug in a program that allows arbitrary
memory to be written to and potentially executed. In most modern languages,
array writes and accesses are checked by bounds checking algorithms to make
sure this doesn't happen, but if these are not written properly, a user could
potentially come up with a way that can be exploited.

When a stack or buffer overflow is done, all bets are off on security. The
attacker could theoretically do anything the program could do.

\end{document}

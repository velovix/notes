\documentclass{article}
\author{Tyler Compton}
\title{Chapter 2: Operating System Structures}

\begin{document}

\maketitle
\tableofcontents

\section{Operating System Services}
Operating systems provide a variety of services to allow for the execution of
programs. These include services that help the user choose what programs they
would like to execute and how they shall execute.

\subsection{Program Execution}
The operating system must be able to load a program into memory, start its
execution, and end execution either normally or in the case of an error.

\subsection{User Interface}
A user interface must be provided for the user to be able to interact with the
operating system and other programs. Modern operating systems provide a
modern graphical or command line interface for this purpose.

\subsubsection{Command Line Interfaces}
As mentioned earlier, operating systems tend to come with a command line
interface where commands can be entered directly. This is sometimes implemented
in the kernel. There can be multiple flavors that an operating system can
provide. Bash is a common one. These are called shells.

\subsubsection{GUI}
Modern systems can run a desktop interface that is more friendly and these days
more common. Users tend to interact with these systems using keyboards, mice,
and almost certainly a monitor. The creator of this form of user interface is
Xerox PARC.

\subsection{I/O Operation}
The operating system must be able to provide I/O operations to programs when
necessary. File I/O is a common task that programs need and the operating
system is in charge of providing.

\subsection{File System Manipulation}
The operating system must provide an interface to interact with files, namely
the ability to create, delete, and edit them. Other operations include looking
in directories and changing permissions.

\subsection{Communications}
Network connectivity to other computers or to processes in the same computer
must be arranged by the operating system.

\subsection{Error Detection}
Errors must be handled in some way.

\subsection{Resource Allocation}
The operating system handles how resources can be accessed. This is especially
important when multiple programs are running at once and trying to access
resources at the same time.

Resources include the CPU, memory, files, or network access.

The operating system must also keep track of how much of each resource a
program is using.

\subsection{Permissions}
The operating system has to be able to protect itself from programs that either
accidentally or purposefully try to trample over it. This is done through
managing permissions, among other methods.

\section{System Calls}
System calls are an interface the operating system provides to programs that
allow them to ask the operating system to do things that the program itself
cannot normally do. System calls are the interface between user space and
kernel space.

These are often not accessed directly by programs. Languages usually expose an
API in the standard library that do the system calls internally.

Each kind of system call is usually assigned a unique number. A table of these
system calls and their numbers is called a system-call interface.

Programs should not have to know anything about how the operating system
achieves the tasks system calls do. They just have to use the provided API
appropriately.

In some circumstances, the kernel needs information from the program before it
can execute a system call. An example would be a call to open a file. The
program must tell the kernel what file it wants to open. This information can
be passed in registers, stored in memory (which is how Linux does it), or put
onto a stack. A nice bonus to the memory block and stack implementations is
that many parameters can be passed at once.

\subsection{Types of System Calls}
\begin{itemize}
	\item Create a process
	\item Abort a process
	\item Execute a command
	\item Managing process attributes
	\item Sleeping
	\item Waiting for events
	\item Memory management
	\item Memory dumps, in case of a critical error
	\item Debugger functionality, like single step execution
	\item Locks for concurrent accesss of resources
	\item File management
	\item Device management
	\item Accessing system data, like time, process info, file info, etc.
	\item Communication
	\item Permissions
\end{itemize}

\section{Example: MS-DOS}
MS-DOS is a relatively primitive operating system. Its operation is very
simple.

When MS-DOS is booted up, it kicks the user to the terminal. Only one program
besides the kernel can be run at once. There's no concept of processes and
there's only one memory space. When a program is executed, it overwrites
everything in memory except for the kernel. When a program is exited, the shell
reloads.

\section{Example: BSD}
BSD is a Unix-like operating system. It is much more complicated and capable
then MS-DOS. It is short for Berkley Software Distribution.

BSD is a multitasking operating system. When the operating system starts, it
will boot the user to the shell, but this shell can be chosen. Processes
do exist, and can be forked off by the shell. The shell continues to run even
as the program runs, and continues after the current program is done running.
Programs can tell the shell whether execution was successful or not.

\section{System Programs}
System programs are programs that interface with system calls in a more
friendly way. Common ones include:

\begin{itemize}
	\item File creation, deletion, reading, writing, etc.
	\item Status information. Date, available memory, disk space, etc.
	\item Programming language support. Compilers, assemblers, etc.
	\item Program execution. Absolute loaders, relocatable loaders,
		linkage, editors, etc.
	\item Communication among processes, users, etc.
	\item Miscelaneous background services that manage various system tasks
\end{itemize}

Programs can interact with these programs instead of doing system calls
directly, which is more convenient.

\section{Implementation}
Operating systems are and have been wrote in a variety of languages. Early
operating systems were written in assembly. Early programming languages like
Alogol started becoming popular. Now, systems languages like C and C++ are most
commonly used.

A modern operating system is generally written in C with key parts in assembly.
Higher level portions are written in any number of languages.

\section{Structure}
Operating systems are very large. They can be structured in a variety of
different ways.

\subsection{Simple}
Early operating systems like MS-DOS provide very little in the way of resource
management or protection. These operating systems aren't common any more, but
had the advantage of giving programs a lot of optimization potential because
they can basically do anything they want.

\subsection{Layered}
Unix operating systems are significantly more complicated. The operating system
is put into layers. User programs live in user space, where a system call
interface is provided to bridge it to kernel space. The kernel is what manages
low level functionality like devices and memory.

\subsection{Microkernel}
Microkernel infrastructure is an attempt to move more kernel functionality into
user space.

Traditional monolithic kernels rely on the quality of modules,
because it shares memory space with them. If someone writes a kernel module
that makes memory errors, the entire operating system could go down.
Microkernels wouldn't have that problem because they give modules their own
memory space.

\end{document}

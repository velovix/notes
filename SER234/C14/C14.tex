\title{Chapter 14: Protection}
\author{Tyler Compton}

\documentclass{article}

\begin{document}

\maketitle
\tableofcontents

\section{Introduction}

The operating system must have some way of making sure resources are accessed
correctly and only by processes that should be allowed to access them. In a
protection model, a computer is a collection of objects, be it hardware or
software.

\section{Principles}

The main guiding principle of protection is the ``principle of least
privilege'', which is the idea that processes and users should be given just the
amount of privileges they need to run and nothing more. This comes from the idea
that all privileges are potential liabilities, so we want to minimize them as
much as possible.

These privileges can be static, meaning that they're the same throughout the
lifespan of the process, or dynamic, meaning that they can be escalated as
needed.

The operating system has to decide what counts as an individual privilage.
Privilages can be rough-grained, meaning that each privilage allows for a wide
variety of operations. Rough-grained privileges are simpler and easier to manage
because there's less to consider, but has the side effect of potentially giving
processes more privilage than they need. Alternatively, privilages can be
fine-grained, meaning that each privilage allows for only a small amount of
additional functionality. This allows processes to be given the smallest amount
of permissions possible, but does complicate matters because there's more to
consider.

\section{Domains}

Domains are a set of access-rights. An access-right contains the name of the
object the access is effecting, and a rights-set, which is a subset of all valid
operations that can be performed within this domain.

In UNIX, domains are attached to user IDs. Files in UNIX have domain bits that
communicate what domains (or users) are able to accomplish certain tasks. A
domain switch is controlled using passwords. A UNIX user uses a root tool like
``su'' along with a root password to temporarily become the root user, which has
a less restrictive access domain.

\section{Access Matrix}

Protection can be visualized as a matrix where the rows are domains and the
columns are objects. Given a domain and an object, one can look at an access
matrix and find what their permissions are.

\begin{tabular}[c c c]
	obj/domain	& $F_1$	& $F_2$ & $F_3$ & printer \\
	$D_1$		& read 	&		& read		& \\
	$D_2$		&		&		&			& print \\
	$D_3$		&		& read	& execute	& \\
	$D_4$		& read write	&		& read write & \\
\end{tabular}

The access matrix can only be changed by someone in a domain with sufficient
permissions. The matrix seperates mechanism from policy. The operating system
maintains the matrix and what rules there are and makes sure only certain
domains can change it, and the user decides what domains have access to what
objects by modifiying this matrix.

\subsection{Implementing Access Matrix}

The most basic implementation is a simple global table with a bunch of data
structures containing information on the domain, the object, and the rights-set.
When someone wants to do an operation on an object, the global table is
consulted to see if they are allowed. However, this solution doesn't work very
well if the access matrix is large. The size of the global table will increase
exponentially with the more combinations of domains and objects that exist. This
implementation is only really good for sparse access matricies.

A better way of implementing an access matrix is to make it a per object list.
Each column contains keypairs of domains and rights-sets for the object of that
column. It is easy to extend this implementation to have a default set, whatever
that means.

Alternatively, the access matrix implementation can be domain-based. Each column
refers to a domain and contains keypairs of objects and the rights-set for the
domain of that column. This keypair collection is known as a capability list,
and each keypair is known as a capability.

Finally, we can make an access matrix implementation that uses locks and keys.
In this system, each object has a list of unique bit patterns, known as locks.
Each domain has a list of unique bit patterns called keys. A process in a domain
can only access an object if one of its keys matches one of the locks. This
makes it easy to express a situation wherein, for instance, a wide variety of
objects are accessable by a domain, because a domain only needs to have one key
to unlock a wide variety of objects.

Every configuration has trade-offs. The global table is very simple to implement
but can get very large very fast. The access list makes it easy to check
permissions at access-time but it becomes non-trivial to figure out the
permissions of a domain, because all objects must be consulted. Plus, the access
list must be consulted for every object access, which can slow things down.
Capability lists make it easy to check the permissions of a domain, but if
someone wants to change or check who can access an object, every domain must be
consulted. Lock-key is a good compromise. Keys can be given from one domain to
another, and it's easy to revoke permissions by making a key not work anymore.

Most systems use a hybrid of access lists and capability lists. The first time
an object is accessed, the access list is consulted. Then, a capatability list
is created that states the process can access that object from now on, which
saves time on accesses. Then, this capatibility list is destroyed when the
process ends.

\section{Revoking Access}

\subsection{Access Lists}

Revoking privilages in an access list system is relatively easy. Just search
through the access list and remove entries that we don't want on there any more.

\subsection{Capability Lists}

In a capability list, every column has to be looked through so that every
instance referring to the object in question must be edited. This can take
longer than with access lists.

\section{Language-Based Protection}

Programming languages can provide high-level protection functionality themselves
to provide more guarantees then the operating system can. This only really
applies to programming languages where a trusted execution environment is
installed on a user's machine that runs untrusted programs.

The JVM provides this protection in a way. Each class has a protection domain
that describes what it can and can't do. If the class attempts to do some kind
of privilaged operation, the JVM will look at the domain the class is in to see
if it should be able to do that operation.

A basic example of this is access rights to protected or private variables or
methods. This is important for object-oriented languages like Java where
encapsulation and data hiding is important.

\end{document}

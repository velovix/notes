\title{Chapter 9: Virtual Memory}
\author{Tyler Compton}

\documentclass{article}

\begin{document}

\maketitle
\tableofcontents

\section{Introduction}
When we run a program, we traditionally load the entire program into memory
before executing. This is fine for small programs, but it's potentially a bit
wasteful. We don't often need the entire program read into memory at once.
Some parts of the code may not be executed during the program's lifecycle, so
memory dedicated to that code is essentially a waste.

Imagine if we could execute a program that's only partially loaded. If that was
possible, then we could fit more programs into memory at once and allow for
more space where other data can go. Programs can be switched between more
quickly because they are more lightweight. This sounds great. How can we do it?

The way we can accomplish this task is by using ``virtual memory''. Virtual
memory is the seperation of logical and physical memory. By allowing these to
be seperate, we can have a logical memory that is much larger than the physical
memory. It also allows for physical address spaces to be shared between
processes. Generally speaking, virtual memory allows the operating system to
get clever with how it places and loads data.

\section{Virtual Address Space}
The virual address space is the logical view of how a process is stored in
memory. It starts at address zero and goes contiguously. Virtual address space
is the abstracted form of how the process is being stored in memory. It can be
simple while the actual way it's stored may be more complicated.

Generally, the Virtual Address Space is mapped so that the stack starts at the
end of the memory space and grows ``down'', while the heap starts after some
static data and grows ``up''. That way, a program can have a large stack and
small heap or vice versa. As the stack and heap grow, more physical memory is
allocated as pages are filled.

As another benefit of abstracting virtual address space from the physical
variant, the OS can have programs share pages with common information, like
shared libraries. However, it still feels to the programs like they have their
own memory domain. Child processes can also share pages, which make process
creation potentially quicker.

It is the job of the MMU to map logical addresses to physical addresses.

\section{Demand Paging}
We've already introduced the idea of loading the program into memory in chunks
instead of all at once, but we can also do that with general data. When
switching to a process, we can save time by not loading all the program memory
in at once. Instead, pages can be brought into memory on demand, meaning that
the page will be brought it when it is required.

A ``lazy swapper'' takes this concept to the extreme and only ever swaps pages
in when they are needed. A swapper that deals with pages is known as a
``pager'' because talent.

With swapping a pager has to guess what pages will be used and brings only
those into memory. We need the MMU to be able to detect what pages are needed.
If a page is ``memory resident'', the behavior is no different from non-
demant paging. However, if the page is not memory resident, the pager needs
to find and load the page into memory without the program having to know that
this happened.

\section{Valid-Invalid Bit}
Every entry in the page table has a valid-invalid bit. ``Valid'' pages are
in memory, or ``memory resident''. ``Invalid'' pages are not in memory.

Pages are initially set to invalid. If the MMU translates a logical address
and finds it points to an invalid page table, then it results in a ``page
fault''

\section{Page Faults}
When a page is referenced and it is marked as invalid, a page fault will be
emitted. The operating system then has to decide if the page is really an
invalid reference (which will result in an error) or if the page just isn't
loaded in memory. If it isn't loaded in memory, the operating system will swap
it into a free frame and update the table to make the page valid. It will
then run the instruction again.

\section{Pure Demand Paging}
The most radical form of demand paging is pure demand paging, which starts a
process with no pages in memory. The very first instruction will result in a
page fault and memory will be swapped to memory and the instruction will be
run again. One could imagine that this is a very slow strategy, because
instructions have to be run at least twice more often than not.

Demand paging requires hardware support. The page table needs to support
valid/invalid bits and there must be a designated swap space somewhere.

Demand paging can be made faster by optimizing how the swap space is stored so
that read and write operations don't take so long. Pages can be discarded
instead of saved back to swap, making writes to swap less frequent.

\section{Copy-on-Write}
Copy-on-Write allows child processes to share all memory with the parent
initially. Copies of memory will only be made if either of the processes
modify the memory. This allows for much quicker process creation and allows for
the least amount of wasted memory possible. A copy-on-write fork can be done
with the vfork syscall on Unix.


Generally, data is copied to a pool of ``zero-fill-on-demand'' pages. Pools
should always have free frames.

\section{Running Out of Free Frames}
If the operating system is running out of free frames, it will start doing
some ``page replacement''. Page replacement is the process of finding pages
that aren't in use and repurposing them. What repurposing them means depends on
the operating system, but it could swap the page out, replace the page, or
terminate the program that owns the page.

Page replacement would be implemented in the page fault service routine, the
algorithm that's called when a page fault happens.

Pages have a ``modify bit'' that is put high when the page is modified. Pages
that haven't been modified won't be rewritten to disk.

The basic page replacement algorithm is as follows:

\begin{itemize}
	\item Find the desired page on disk
	\item Find a free frame, or use page replacement to find a ``victim
		frame''. The victim frame will be written to disk if it's been
		modified
	\item Swap the page into the frame in memory
	\item Update the frame table
	\item Restart the instruction that caused the page fault
\end{itemize}

\end{document}

\title{Chapter 7: Deadlocks}
\author{Tyler Compton}

\documentclass{article}

\begin{document}

\maketitle
\tableofcontents

\section{Introduction}
When sharing resources, software takes on a general pattern of requesting the
resource, using the resource, and releasing the resource to be used by someone
else. Deadlocks can happen if the data is mutually exclusive, meaning only one
process can use it at once, the process holding one resource is waiting for
another resource, resources can only be released when the process decides it's
finished, and there exists a circular wait, wherein processes end up each
waiting on one another. No process will ever release the resource the other
process is waiting on, so the program will never be able to continue. This is
obviously something we want to avoid.

\section{Resource-Allocation Graph}
A resource-allocation graph is a graph showing the relationships between
processes and the resources they request. This is a good way to inspect for
possible deadlocks.

Resource-allocation graphs are made of verticies and nodes. Verticies are like
points, and they represent either processes, represented by a circle with a P,
or resources, which are cubes with an R under them. Nodes come in two types, as
well. There are request edges, which represent a request from a process to a
resource, and there are assignment edges, which represent a process holding an
instance of a request, which point the opposite way of a request edge. Resource
nodes can have multiple instances inside of them, which are represented by dots
or squares inside of a resource.

Reading these graphs can betray deadlocks in a software system by looking for
cycles in the graph. A cycle is a scenario wherein a process owns a resource
and requests a resource which is owned by a process that owns the resource the
first process wants. In general, if a graph contains no cycles, there is no
possibility of deadlocks. If the graph contains a cycle and there is only one
instance of a resource in that deadlock, then there is definitely a deadlock.
If there are several instances of the resource types, there may be a deadlock.

\section{Handling Deadlocks}
Operating systems can choose to handle deadlocks in a variety of ways. The OS
can choose to try to prevent deadlocks in some way, make attempts to avoid
deadlocks in some way, let deadlocks happen but provide a way to recover, or
pretend that deadlocks don't happen and make no effort to achieve that. Many
operating systems use the latter strategy.

\section{Deadlock Prevention}
Deadlocks can be prevented in a variety of ways. The operating system can
avoid mutual exclusion by having read-only resources. Ideally, if no resources
have to be mutually exclusive, no deadlocks can occur.

The operating system can use ``Hold and Wait'', which requires processes to
not be holding on to other resources when it requests a new one. This makes
deadlocks impossible, because no cycles can be formed.

The operating system can use ``No Preemption'', meaning that if a proccess is
holding a resource and it attempts to get another, the OS will check if the
requested resource can be immediately given to the process. If it cannot, then
the operating system will release all resource from the process and add those
resources to the list of waiting resources. The process will be started again
only when all those resources can be allocated to the process at once. This can
completely prevent deadlocks because it stops a potential circular resource
dependency before it can happen.

The ``Circular Wait'' system dictates a total ordering of all resource types,
and requires processes to request resources in an increasing order of
enumeration.

\section{Deadlock Avoidance}
Deadlock avoidance requires some added information from processes in order to
make intelligent decisions about resource allocation. The operating system may
ask what the maximum number of resources of each type the process will need,
then it may examine the resource allocation state intelligently to try to avoid
deadlocks. The ``resource allocation state'' is the number of available and
allocated resources, and the maximum demands of each process.

The operating system can deem that it is in a ``safe state'' so long as there
exists a sequence of all processes running in the system such that for each
process $P_i$, resource requests can be satisfied by the currently available
resources or the resources held by a process $P_j$, where $j$ is less than
$i$. This creates a hierarchy of resources such that processes after a given
process will not ever have to make requests to a resource owned by that
process. If the operating system is in a safe state, there is no possibility of
deadlocks. If the operating system makes sure that no unsafe states can be
entered, there will never be a deadlock.

\subsection{Extended Resource-Allocation Graph Scheme}
In order to make a resource-allocation graph that works with a deadlock
avoidance scheme, we need some more grammar. A ``claim edge'' is an edge that
means a process might make a request to a resource. It is shown using a dotted
line.

During runtime, a resource-allocation graph morphes to reflect the changing
state of the system. A claim edge becomes a request edge when a process
actually makes the request. A request edge becomes an assignment edge when the
process actually gets the resource. When that resource is released, the
resource edge is converted back to a claim edge.

\end{document}

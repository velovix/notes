\documentclass{article}
\author{Tyler Compton}
\title{Chapter 4 - Linked Structures - Stacks}

\usepackage{float}
\usepackage{amsmath}
\usepackage{listings}

\usepackage{color}

\definecolor{codebackgroundcolor}{rgb}{0.95,0.95,0.95}
\definecolor{codecomment}{rgb}{0,0.6,0}
\definecolor{codenumber}{rgb}{0.5,0.5,0.5}
\definecolor{codestring}{rgb}{0.58,0,0.82}
\lstdefinestyle{codestyle}{
	backgroundcolor=\color{codebackgroundcolor},
	commentstyle=\color{codecomment},
	keywordstyle=\color{magenta},
	numberstyle=\color{codenumber},
	stringstyle=\color{codestring},
	basicstyle=\footnotesize,
	breakatwhitespace=false,
	breaklines=false,
	captionpos=b,
	keepspaces=false,
	numbers=left,
	numbersep=5pt,
	showspaces=false,
	showstringspaces=false,
	showtabs=false,
	tabsize=4
}
\lstset{style=codestyle} 

\lstset{style=codestyle}

\begin{document}

\maketitle
\tableofcontents

\section{Introduction}
The standard way to store many objects is with a traditional array. Arrays need
to be stored in a memory block so that they can be reliably indexed. The
programming language knows that it must traverse $size(obj)*i$ bytes before
finding the target. This makes indexing very fast, but it means that arrays
have to allocate in large blocks. This encourages memory fragmentation and
makes arrays relatively inflexible, size-wise.

Linked lists are an alternate way to deal with this problem that fixes a lot of
the issues that come with arrays. Linked lists are made up of individual
objects that hold references to another object in the list. This creates a
daisychain of objects. These are called nodes. Since objects are treated
individually, there's no need to allocate large blocks of memory. It's also very
easy and cheap to append elements to the end of the list by simply creating it
and having the last element point to this new object. Deleting elements is also
very simple.

Linked lists have the disadvantage of making indexing much more expensive. The
program has to follow this daisychain in order to find the object at an index
instead of simply jumping there.

\section{Constructing Linked Lists}
Oftentimes, we'll want to construct linked lists out of objects that weren't
designed to be in a linked list, so they won't have the place to store
references to neighbor objects. We can solve this problem by creating a node
class that contains the object and references to neighbor objects.

\section{Doubly Linked Lists}
Doubly linked lists contain two references in their nodes: a reference to the
next object and a reference to the previous one. This makes traversing
backwards in the list possible. This isn't useful for a stack, but one could
imagine a situation where this would be helpful.

\section{Cyclicly Linked Lists}
Cyclicly linked lists are lists whose last element links to the first element.
Singly and doubly linked lists can also by cyclicly linked lists.

\section{Linked List Stacks}
Implementing a stack using links is fairly straightforward. Because all we're
concerned about is the top of the stack, we only need to hold a reference to
what's on top. The list can be singly linked to the object before it so that we
can pop off the stack easily.

\end{document}

\documentclass{article}
\author{Tyler Compton}
\title{Chapter 3 - Introduction to Collections}

\usepackage{float}
\usepackage{amsmath}
\usepackage{listings}

\usepackage{color}

\definecolor{codebackgroundcolor}{rgb}{0.95,0.95,0.95}
\definecolor{codecomment}{rgb}{0,0.6,0}
\definecolor{codenumber}{rgb}{0.5,0.5,0.5}
\definecolor{codestring}{rgb}{0.58,0,0.82}
\lstdefinestyle{codestyle}{
	backgroundcolor=\color{codebackgroundcolor},
	commentstyle=\color{codecomment},
	keywordstyle=\color{magenta},
	numberstyle=\color{codenumber},
	stringstyle=\color{codestring},
	basicstyle=\footnotesize,
	breakatwhitespace=false,
	breaklines=false,
	captionpos=b,
	keepspaces=false,
	numbers=left,
	numbersep=5pt,
	showspaces=false,
	showstringspaces=false,
	showtabs=false,
	tabsize=4
}
\lstset{style=codestyle} 

\lstset{style=codestyle}

\begin{document}

\maketitle
\tableofcontents

\section{Introduction}
Collections are objects that hold and organize other objects. Collections have
methods and other tooling for accessing and manipulating their elements.

There are quite a few collections that have been defined in the Java standard
library. Collections are usually in charge of maintaining a form of
relationship between elements. Some of them are linear, meaning that they have
some kind of order where every element is after and/or before another element.
Others take on a more complicated relationship, like a tree. Different kinds of
collections have many attributes that make them unique.

\section{Abstraction}
Abstractions hide the details to make a concept easier to manage. Every object
is an abstraction in that it provides an interface to interact with the object.
Other parts of the program that use the object do not need to know exactly how
the methods it uses are implemented, only that it does what is advertised.

Many objects engage in information hiding, in that direct access to its
variables is not allowed outside of the object itself. Access is done through
methods, which can have any amount of AOP elements inside of them.

\subsection{Abstract Data Types}
A data type, at least in object-oriented programming terms, is a group of
values and behavior that operates on those values. These come in the form of
variables and methods.

Abstract data types are not language builtins and represent some higher level
concept. When you use abstract data types, you're dealing with a concept that
isn't tied to the programming language. When we work with an implementation of
a stack in Java, we're thinking about it in terms of a stack, not a language
primitive.

A data structure is a set of programming language constructs and techniques to
implement a collection.

All these terms are easily confused and are often used interchangably in casual
language.

\section{Stacks}
Stacks are simple collections that provide a minimal interface for storing
data. Stacks are metaphors for stacks in real life. When you add an element,
that element is added to the top of the stack. The only way you can access
elements is by taking them off the top of the stack.

Stacks are "last in, first out" (LIFO) data structures, meaning that after
putting in a certain amount of elements, the last element to be put in will be
the first accessible one, because it is at the top of the stack.

When an element is put on to the stack, we say we are ``pushing'' that element
onto the stack. When an element is taken from the top, we call it ``popping``
the element off the stack. Most implementations also allow users to peek at the
top of the stack without taking it off the stack. They also generally allow the
user to check the size of the stack, or at the very least check to see if it's
empty.

\section{Generics}
Writing code in Java that can operate on multiple types used to involve
interacting with the Object type. Unfortunately, this required a lot of runtime
type checking in the form of casting in order to stay in control of what types
are allowed to be used.

The new way to do this is to use Java's native support for generic typing.
Generic classes and methods can be written that refer to the type being used
generically. The type is specified automatically at runtime depending on how
the user uses the generic class.

\section{Capacity}
Capacity is the amount of elements a container can store. In Java, there's no
way to change the capacity of an array. Java will instead create a new array
with the new capacity. In Java, containers have optimizations that make this
happen less frequently by requesting more space than you actually need.

\end{document}

\title{Chapter 8: Recursion}
\author{Tyler Compton}

\documentclass{article}

\begin{document}

\maketitle
\tableofcontents

\section{Introduction}
A recursive function is a function that calls itself. That's all there is to
it. Recursion is very powerful and can solve many of the problems we use loops
to solve.

\section{Lists}
Lists can be defined recursively by being thought of as an element at the
beginning and a list of everything else at the end, or just the first element
if nothing else exists. This definition of a list uses the word ``list'' inside
of the definition, so it is recursive. The situation in which a list is just
one element is the definition's ``base case''.

\section{Infinite Recursion}
Recursive definitions must have a non-recursive portion, which is oftentimes
the base case. If the definition does not have a non-recursive part, the
recursive definition will never end. However, this is oftentimes less desirable
than an infinite loop, because an infinite recursion will often run the program
out of stack frames and it will crash.

\section{Recursion in Math}
Recursion is often used in mathematics. For instance, the factorial definition
is defined using another factorial, which is defined using another factorial
and so on until the base case of 1 is reached. Using recursion in programming
can allow us to reflect these definitions elegantly.

\section{Recursion in Java}
Recursion can be formed in Java by having a method call itself. This is known
as a ``recursive method''. The body of the recursive method must contain code
for every case, including the recursive and non-recursive cases. This is often
in the form of a chain of if-else statements.

In Java, each call to the method sets up a new scope with new parameters and
local variables. When the method completes, it returns to the place that
executed the method.

\section{Recursion vs. Iteration}
Theoretically, recursion can solve any solution iteration can solve, and vice
versa, but just because we can use recursion doesn't necessarily mean that we
should. Oftentimes, iteration can be easier to follow and is what most
programmers are used to seeing.

Recursion is also associated with a lot of overhead. Iterative solutions don't
create a new stack frame for every iteration, for instance.

There are situations where recursion solves a problem very elegantly, but in a
non-purely-functional language like Java, it's best used sparingly.

\section{Types of Recursion}
The recursion we commonly think of is known as ``direct recursion''. A recursive
definition is direct when a function directly calls itself. The other type of
recursion is known as ``indirect recursion'', wherein a chain of function calls
eventually loops on itself, but no functions call themselves directly.

\section{Analyzing Recursive Algorithms}
To find the order of a loop, we multiply the order of the body of the loop with
the amount of loop executions. We can do the same thing with recursive
definitions by multiplying how often the recursive function runs by the order
of the function.

\end{document}

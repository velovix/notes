\documentclass{article}
\author{Tyler Compton}
\title{Chapter 6 - Lists}

\usepackage{float}
\usepackage{amsmath}
\usepackage{listings}

\usepackage{color}

\definecolor{codebackgroundcolor}{rgb}{0.95,0.95,0.95}
\definecolor{codecomment}{rgb}{0,0.6,0}
\definecolor{codenumber}{rgb}{0.5,0.5,0.5}
\definecolor{codestring}{rgb}{0.58,0,0.82}
\lstdefinestyle{codestyle}{
	backgroundcolor=\color{codebackgroundcolor},
	commentstyle=\color{codecomment},
	keywordstyle=\color{magenta},
	numberstyle=\color{codenumber},
	stringstyle=\color{codestring},
	basicstyle=\footnotesize,
	breakatwhitespace=false,
	breaklines=false,
	captionpos=b,
	keepspaces=false,
	numbers=left,
	numbersep=5pt,
	showspaces=false,
	showstringspaces=false,
	showtabs=false,
	tabsize=4
}
\lstset{style=codestyle} 

\lstset{style=codestyle}

\begin{document}

\maketitle
\tableofcontents

\section{Introduction}
A list is another type of collection like stacks or queues. It is a linear
collection in that it is one-dimensional, order-wise. Lists are very generic
and flexible. You can add, access, and remove from anywhere in the list at any
time.

\section{Operations}
Queues have a standard set of operations.

\begin{itemize}
	\item enqueue: Add an item to the queue
	\item dequeue: Remove an item from the end of the queue
	\item front: Examine the item on the end of the queue
	\item isEmpty: Pretty self-explanitory
	\item size: The amount of items on the queue
	\item toString: I don't think this actually counts, but whatever
\end{itemize}

\section{Ordered Lists}
Ordered lists are lists whose elements are always ordered by some
characteristic of the elements, like a high score or alphabetically.

\section{Unordered Lists}
Ordered lists, unlike what their name implies, does in fact have order, but
the order is decided by the user. If the user wants to put a new element at the
beginning of the list, the user may do that.

\section{Indexed Lists}
Indexed lists are like ordered lists, but has the added feature that elements
are accessed using their index on the list. If an element is added before other
elements, all the proceeding elements will be updated. For that reason, it is
not necessarily safe to assume that an element is always tied to a given index.

In Java, the List interface specifies the interface for an indexed lists. This
allows you to index a list backed by a linked data structure, which isn't very
efficient.

\end{document}

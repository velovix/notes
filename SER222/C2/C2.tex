\documentclass{article}
\author{Tyler Compton}
\title{Chapter 2 - Analysis of Algorithms}

\usepackage{float}
\usepackage{amsmath}
\usepackage{listings}

\usepackage{color}

\definecolor{codebackgroundcolor}{rgb}{0.95,0.95,0.95}
\definecolor{codecomment}{rgb}{0,0.6,0}
\definecolor{codenumber}{rgb}{0.5,0.5,0.5}
\definecolor{codestring}{rgb}{0.58,0,0.82}
\lstdefinestyle{codestyle}{
	backgroundcolor=\color{codebackgroundcolor},
	commentstyle=\color{codecomment},
	keywordstyle=\color{magenta},
	numberstyle=\color{codenumber},
	stringstyle=\color{codestring},
	basicstyle=\footnotesize,
	breakatwhitespace=false,
	breaklines=false,
	captionpos=b,
	keepspaces=false,
	numbers=left,
	numbersep=5pt,
	showspaces=false,
	showstringspaces=false,
	showtabs=false,
	tabsize=4
}
\lstset{style=codestyle} 

\lstset{style=codestyle}

\begin{document}

\maketitle
\tableofcontents

\section{Introduction}
Software engineers have a vast array of tools to solve problems. These tools
are referred to as algorithms. Naturally, we want to choose the algorithm that
solves the problem most efficiently. The efficiency of an algorithm is usually
expressed in terms of CPU time, but CPU time is platform specific. We can come
up with formulas that express an algorithm's efficiency.

\section{Algorithm Efficiency}
As an example, consider a dish washer that needs to redry all previously washed
dishes after washing one. It takes 30 seconds to wash one dish, and 30 seconds
to dry a dish. The formula for this algorithm would be the following.

$$time(ndishes) = 30n + \sum_{i=1}^{n}(i+30)$$

$$time(ndishes) = 30n + \dfrac{30n(n+1)}{2}$$

\section{Problem Size}
When choosing algorithms, we need to know the size of our problem. In the above
example, the size is the amount of dishes that need washing. For a sorting
algorithm, it is the amount of items that will be searched through.

\section{Growth Functions}
We also must consider what resources we need to optimize our use of most.
Usually, we target CPU time, but on memory-starved system, we may be concerned
about RAM, instead.

A growth function is the relationship between the size of the problem and the
value that we are concerned about. Growth functions can be very useful when
comparing algorithms that are similar in other ways. The following is the
growth function of the dish washer.

$$t(n) = 15n^2 + 45n$$

\section{Asymptotic Complexity}
Asymptotic complexity is a simplified growth function. It concentrates entirely
on the dominate term. In the case of the dish washer growth function, the term
$15n^2$ grows far, far faster than $45n$. As a result, the more our problem
size increases, the less important the factor $45n$ becomes. Similarly, the
fact that the term $15n^2$ is multiplied by 15 becomes more and more
insignificant as the problem size increases, as well. Therefore, the asymptotic
complexity of the dish washer is $n^2$.

When compared to their original growth functions, an algorithm's asymptotic
complexity should either be slower (as in it grows faster) or equivalent.

\subsection{A Note On Log}
In mathematics, logarithmic functions tend to be base 10 unless otherwise
specified. In science, logarithmic functions tend to be base $e$. In computing,
logarithmic functions tend to be base 2.

This seems frustrating, but remember that, for asymptotic complexity,
$log_10 (n)$ is equivalent to $log_2 (n)$. As a result, it hardly matters.

\section{Big-Oh Notation}
Big-Oh notation is a standard form for asymptotic complexity. For the dish
washing algorithm, it's big-oh notation is $O(n^2)$. Because asymptotic
complexity simplifies growth functions so much, algorithms begin to fall into
categories. Algorithms that may have different growth functions but the same
asymptotic complexity are generally treated as if they are the same complexity.

\begin{table}[H]
	\centering
	\begin{tabular} { l l l }
		Growth Function & Order & Label \\
		$t(n) = 17$ & O(1) & constant \\
		$t(n) = 3log(n)$ & O($log(n)$) & logarithmic \\
		$t(n) = 20n - 4$ & O(n) & linear \\
		$t(n) = 12n log(n) + 100n$ & O($n log(n)$) & n log n \\
		$t(n) = 3n^2 + 5n - 2$ & O($n^2$) & quadratic \\
		$t(n) = 8n^3 + 3n^2$ & O($n^3$) & cubic \\
		$t(n) = 2^n + 18n^2$ & O($2^n$) & exponential \\
	\end{tabular}
\end{table}

\section{Why Does Complexity Matter?}
For simple problems, computers often feel nearly infinitely fast. When you're
not dealing with huge sets, it often feels like algorithm efficiency hardly
matters. Won't Moore's Law take care of it? Nope.

Imagine if we manage to upgrade our processor so that it is ten times as fast.
It would seem like our algorithms would get way faster and we could afford to
use less efficient algorithms where we used to have to be stingy. However,
the benefit is oftentimes much less significant than one might hope

\begin{table}[H]
	\centering
	\begin{tabular}{c c c c}
		Algorithm & Time Complexity & Orig. Max Size & New Max Size \\
		A & n     & $s_1$ & $10 s_1$    \\
		B & $n^2$ & $s_2$ & $3.16 s_2$  \\
		C & $n^3$ & $s_3$ & $2.15 s_3$  \\
		D & $2^n$ & $s_4$ & $s_4 + 3.3$ \\
	\end{tabular}
\end{table}

As you can see by the above table, the increase in speed is only linear for
algorithms that grow linearly. The speed increase gets less significant as the
complexity of the algorithm increases. In fact, for exponential functions like
Algorithm D, the change is almost insignificant. It takes Moore's Law around
six years to deliver a processor to your door that's ten times faster, and it's
all for nothing when we use inefficient algorithms.

\section{Analyzing Execution}
We oftentimes want to analyze the complexity of a chunk of code. We do that by
counting operations in a worse-case scenario.

Here is an example of a basic algorithm

\begin{lstlisting}[language=Java]
for (int count = 0; count < n; count++) {
	// Some sequence of O(1) steps
}
\end{lstlisting}

We see that we're executing at least one $O(1)$ operation for every element we
have. As a result, the amount of operations we will be executing is directly
and linearly related to the amount of elements. So, our asymptotic complexity
is $O(n)$.

\begin{lstlisting}[language=Java]
int count = 1;
while (count < n) {
	count *= 2;
	// Some sequence of O(1) steps
}
\end{lstlisting}

In this case, we're still executing $O(1)$ operations based on how many
elements we have, but it isn't a linear relationship anymore because count is
increasing faster overall then in the previous example. As a result, our
asymptotic complexity is $O(log(n))$.

\begin{lstlisting}[language=Java]
for (int count = 0; count < n; count++) {
	for (int count2 = 0; count2 < n; count2++) {
		// Some sequence of O(1) steps
	}
}
\end{lstlisting}

Now, there are two loops, one nested in the other. If it wasn't for the second
loop, we would again be executing $n*c$ (where c is the amount of O(1) steps)
operations. But, in this example we're doing $n*c$ O(1) operations for every
element. To put it simply, we're executing $n*n*c$ O(1) operations, making our
asymptotic complexity equal to $O(n^2)$.

\end{document}

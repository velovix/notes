\documentclass{article}
\author{Tyler Compton}
\title{Chapter 5 - Queues}

\usepackage{float}
\usepackage{amsmath}
\usepackage{listings}

\usepackage{color}

\definecolor{codebackgroundcolor}{rgb}{0.95,0.95,0.95}
\definecolor{codecomment}{rgb}{0,0.6,0}
\definecolor{codenumber}{rgb}{0.5,0.5,0.5}
\definecolor{codestring}{rgb}{0.58,0,0.82}
\lstdefinestyle{codestyle}{
	backgroundcolor=\color{codebackgroundcolor},
	commentstyle=\color{codecomment},
	keywordstyle=\color{magenta},
	numberstyle=\color{codenumber},
	stringstyle=\color{codestring},
	basicstyle=\footnotesize,
	breakatwhitespace=false,
	breaklines=false,
	captionpos=b,
	keepspaces=false,
	numbers=left,
	numbersep=5pt,
	showspaces=false,
	showstringspaces=false,
	showtabs=false,
	tabsize=4
}
\lstset{style=codestyle} 

\lstset{style=codestyle}

\begin{document}

\maketitle
\tableofcontents

\section{Introduction}
A queue is a collection whose elements are added on one end and removed on
another. This creates a first in, first out system, which preserves order.
Grocery lines, cars at a stop light, and assembly lines are all examples of
queues in real life.

\section{Operations}
Queues have a standard set of operations.

\begin{itemize}
	\item enqueue: Add an item to the queue
	\item dequeue: Remove an item from the end of the queue
	\item front: Examine the item on the end of the queue
	\item isEmpty: Pretty self-explanitory
	\item size: The amount of items on the queue
	\item toString: I don't think this actually counts, but whatever
\end{itemize}

\section{Queue Implementations}
A queue can be implemented as a linked list by keeping track of the queue head
node (the value about to be dequeued) and the end node (the value that was just
queued). This is unlike a general linked list implementation in that we store
two nodes, but if we didn't do it that way, we'd have to use a doubly linked
list and would have to traverse the list for every operation.

Queues can also be implemented in an array, but doing so can be kind of tricky
because we want to avoid moving every element over one when queueing an
element. To avoid this, we construct a circular array. Circular arrays are
arrays that conceptually loop back on themselves. If the length of a circular
array is 10, then element 9 would preceed element 0. We have two integers that
keep track of the index of the front and end of the queue. Since the array is a
circle, it shouldn't matter what element we start at. A correct implementation
would allow the user to push 'n' elements into the queue, where n is the size
of the circular array, regardless of what index we start at. Expanding the size
of an array is best done by simply making a new array and copying data.

\end{document}
